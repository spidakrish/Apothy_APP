\documentclass[11pt,a4paper]{article}

% Packages
\usepackage[utf8]{inputenc}
\usepackage[T1]{fontenc}
\usepackage{geometry}
\usepackage{hyperref}
\usepackage{xcolor}
\usepackage{titlesec}
\usepackage{enumitem}
\usepackage{longtable}
\usepackage{booktabs}
\usepackage{fancyhdr}
\usepackage{lastpage}
\usepackage{soul}

% Page geometry
\geometry{margin=2.5cm}

% Colours
\definecolor{apothypurple}{HTML}{8B5CF6}
\definecolor{darkbg}{HTML}{0A0A0F}

% Hyperlinks
\hypersetup{
    colorlinks=true,
    linkcolor=apothypurple,
    urlcolor=apothypurple
}

% Header/Footer
\pagestyle{fancy}
\fancyhf{}
\fancyhead[L]{\textbf{Apothy App Development Log}}
\fancyhead[R]{Apothyai Pty Limited}
\fancyfoot[C]{Page \thepage\ of \pageref{LastPage}}
\renewcommand{\headrulewidth}{0.4pt}
\renewcommand{\footrulewidth}{0.4pt}

% Title formatting
\titleformat{\section}{\Large\bfseries\color{apothypurple}}{\thesection}{1em}{}
\titleformat{\subsection}{\large\bfseries}{\thesubsection}{1em}{}

\begin{document}

% Title Page
\begin{titlepage}
    \centering
    \vspace*{3cm}

    {\Huge\bfseries\color{apothypurple} Apothy App\\[0.5cm] Development Work Log}

    \vspace{1cm}

    {\Large Apothyai Pty Limited}

    \vspace{2cm}

    {\large
    \begin{tabular}{ll}
        \textbf{Project:} & Apothy AI Companion App \\
        \textbf{Platform:} & iOS (Flutter) \\
        \textbf{Repository:} & \url{https://github.com/spidakrish/Apothy_APP} \\
        \textbf{Started:} & 7 December 2024 \\
        \textbf{Bundle ID:} & com.apothyai.apothy \\
    \end{tabular}
    }

    \vfill

    {\large\textit{Born from light. Trained in truth. Built to become what you need.}}

    \vspace{1cm}

\end{titlepage}

\tableofcontents
\newpage

% ============================================================================
\section{Project Overview}
% ============================================================================

\subsection{Technical Stack}
\begin{itemize}
    \item \textbf{Framework:} Flutter 3.38.4
    \item \textbf{Language:} Dart 3.10.3
    \item \textbf{State Management:} Riverpod
    \item \textbf{Local Storage:} Hive
    \item \textbf{Backend API:} AWS Lambda (Python/FastAPI)
    \item \textbf{Authentication:} Apple Sign-In, Google Sign-In, Email
    \item \textbf{IDE:} VS Code / Xcode 16.3
\end{itemize}

\subsection{Design Specifications}
\begin{itemize}
    \item \textbf{Theme:} Dark mode with purple accents
    \item \textbf{Primary Colour:} \texttt{\#8B5CF6} (Purple/Violet)
    \item \textbf{Background:} \texttt{\#0A0A0F} (Near black)
    \item \textbf{Font:} Roboto
    \item \textbf{Navigation:} Bottom tab bar (5 tabs)
\end{itemize}

\subsection{Core Screens}
\begin{enumerate}
    \item Mirror (Welcome/Landing)
    \item History (Chat history with date picker)
    \item Chat (Main conversation interface)
    \item Dashboard (Points, achievements, progress)
    \item Settings (Profile, appearance, notifications)
\end{enumerate}

% ============================================================================
\section{Development Log}
% ============================================================================

\subsection{7 December 2024}

\subsubsection{Session 1: Project Initialisation}
\begin{longtable}{p{3cm}p{11cm}}
\toprule
\textbf{Time} & \textbf{Activity} \\
\midrule
\endfirsthead
\midrule
\textbf{Time} & \textbf{Activity} \\
\midrule
\endhead
Start & Installed CocoaPods via Homebrew \\
& Initialised Git repository \\
& Connected to GitHub: \url{https://github.com/spidakrish/Apothy_APP} \\
& Created Flutter project with bundle ID \texttt{com.apothyai.apothy} \\
& Verified iOS build (Xcode 16.3) \\
& Initial commit and push to GitHub \\
\bottomrule
\end{longtable}

\textbf{Outcome:} Project scaffold complete. iOS build verified working.

\subsubsection{Session 2: Phase 1 --- Foundation}
\begin{longtable}{p{3cm}p{11cm}}
\toprule
\textbf{Time} & \textbf{Activity} \\
\midrule
\endfirsthead
\midrule
\textbf{Time} & \textbf{Activity} \\
\midrule
\endhead
Start & Created work log document (this file) \\
& Set up folder structure (clean architecture): \\
& \quad - \texttt{lib/core/} (constants, theme, utils, services) \\
& \quad - \texttt{lib/features/} (auth, chat, mirror, history, dashboard, settings) \\
& \quad - \texttt{lib/shared/} (widgets, providers) \\
& Added dependencies to \texttt{pubspec.yaml}: \\
& \quad - flutter\_riverpod, riverpod\_annotation (state management) \\
& \quad - go\_router (navigation) \\
& \quad - dio, http (networking) \\
& \quad - sign\_in\_with\_apple, google\_sign\_in (authentication) \\
& \quad - hive, hive\_flutter, flutter\_secure\_storage (storage) \\
& \quad - google\_fonts, flutter\_svg, cached\_network\_image (UI) \\
& Created theme constants: \\
& \quad - \texttt{app\_colors.dart} --- Full colour palette \\
& \quad - \texttt{app\_typography.dart} --- Roboto font styles \\
& \quad - \texttt{app\_theme.dart} --- Material3 ThemeData \\
& Created API and app constants: \\
& \quad - \texttt{api\_constants.dart} --- Endpoint URLs \\
& \quad - \texttt{app\_constants.dart} --- App-wide constants \\
& Created reusable widgets: \\
& \quad - \texttt{app\_button.dart} --- Primary/outlined/text variants \\
& \quad - \texttt{app\_text\_field.dart} --- Input fields \& chat input \\
& \quad - \texttt{message\_bubble.dart} --- User/assistant chat bubbles \\
& \quad - \texttt{gradient\_background.dart} --- Purple glow backgrounds \\
& \quad - \texttt{loading\_indicator.dart} --- Loaders \& skeletons \\
\bottomrule
\end{longtable}

\textbf{Outcome:} Phase 1 (Foundation) complete. All theme constants, barrel files, and reusable widgets created.

% ============================================================================
\section{Architecture Decisions}
% ============================================================================

\subsection{Folder Structure}
The project follows a feature-first clean architecture:

\begin{verbatim}
lib/
  core/           # Shared utilities, constants, themes
  features/       # Feature modules (auth, chat, etc.)
  shared/         # Shared widgets and components
  main.dart       # App entry point
\end{verbatim}

\subsection{State Management}
Using Riverpod for state management because:
\begin{itemize}
    \item Compile-time safety
    \item Provider scoping and overriding
    \item No BuildContext dependency for reading state
    \item Excellent testing support
\end{itemize}

% ============================================================================
\section{API Integration}
% ============================================================================

\subsection{Endpoints}
\begin{longtable}{p{4cm}p{10cm}}
\toprule
\textbf{Environment} & \textbf{URL} \\
\midrule
Production & \texttt{https://xnfhbfcbpptm6bqdvs5gi6gimu0xiguq.lambda-url.ap-southeast-2.on.aws} \\
Staging & \texttt{https://api-staging.apothy.ai} \\
Local & \texttt{http://localhost:8000} \\
\bottomrule
\end{longtable}

% ============================================================================
\section{Issues and Resolutions}
% ============================================================================

\subsection{Issue 1: Deprecated \texttt{withOpacity} API}
\textbf{Problem:} Flutter 3.27+ deprecated \texttt{Color.withOpacity()} in favour of \texttt{Color.withValues(alpha: x)}.

\textbf{Solution:} Updated all 8 occurrences across 6 files to use the new API.

\textbf{Files affected:}
\begin{itemize}
    \item \texttt{lib/core/theme/app\_colors.dart}
    \item \texttt{lib/core/theme/app\_theme.dart}
    \item \texttt{lib/shared/widgets/app\_button.dart}
    \item \texttt{lib/shared/widgets/message\_bubble.dart}
    \item \texttt{lib/shared/widgets/gradient\_background.dart}
    \item \texttt{lib/shared/widgets/loading\_indicator.dart}
\end{itemize}

\subsection{Issue 2: Theme class naming changes}
\textbf{Problem:} Flutter 3.38 renamed \texttt{CardTheme} to \texttt{CardThemeData} and \texttt{DialogTheme} to \texttt{DialogThemeData}.

\textbf{Solution:} Updated \texttt{lib/core/theme/app\_theme.dart} to use the new class names.

\subsection{Issue 3: Dangling library doc comments}
\textbf{Problem:} Barrel files with doc comments but no \texttt{library;} directive triggered linter warnings.

\textbf{Solution:} Added \texttt{library;} directive to all 5 barrel files.

\subsection{Issue 4: Dependency version conflict}
\textbf{Problem:} \texttt{riverpod\_generator \^{}2.6.2} conflicted with \texttt{hive\_generator \^{}2.0.1} due to incompatible \texttt{analyzer} requirements.

\textbf{Solution:} Downgraded \texttt{riverpod\_generator} to \texttt{\^{}2.4.0} as recommended by the Flutter tooling.

\subsubsection{Session 3: Phase 2 --- Navigation Shell}
\begin{longtable}{p{3cm}p{11cm}}
\toprule
\textbf{Time} & \textbf{Activity} \\
\midrule
\endfirsthead
\midrule
\textbf{Time} & \textbf{Activity} \\
\midrule
\endhead
Start & Researched \texttt{go\_router} \texttt{StatefulShellRoute} pattern \\
& Verified implementation approach from official documentation \\
& Created router configuration: \\
& \quad - \texttt{lib/core/router/app\_router.dart} --- Route definitions \\
& \quad - \texttt{lib/core/router/navigation\_shell.dart} --- Bottom nav bar \\
& \quad - \texttt{lib/core/router/router.dart} --- Barrel file \\
& Created placeholder screens for all 5 tabs: \\
& \quad - \texttt{mirror\_screen.dart} --- Welcome/landing page \\
& \quad - \texttt{history\_screen.dart} --- Chat history \\
& \quad - \texttt{chat\_screen.dart} --- Main conversation interface \\
& \quad - \texttt{dashboard\_screen.dart} --- Points and achievements \\
& \quad - \texttt{settings\_screen.dart} --- App preferences \\
& Updated \texttt{main.dart}: \\
& \quad - Added Riverpod \texttt{ProviderScope} \\
& \quad - Integrated \texttt{MaterialApp.router} with \texttt{appRouter} \\
& \quad - Set system UI overlay styles for dark theme \\
& Updated test file to work with new structure \\
& Verified iOS build succeeded \\
\bottomrule
\end{longtable}

\textbf{Outcome:} Phase 2 (Navigation Shell) complete. App builds and runs with 5-tab bottom navigation.

% ============================================================================
\section{Future Work}
% ============================================================================

\begin{enumerate}
    \item \sout{Phase 1: Foundation (theme, constants, widgets)} --- Complete
    \item \sout{Phase 2: Navigation shell (bottom tab bar)} --- Complete
    \item Phase 3: Authentication (Apple, Google, Email)
    \item Phase 4: Core chat experience
    \item Phase 5: Supporting screens (Mirror, History, Dashboard, Settings)
    \item Phase 6: Testing and polish
    \item Phase 7: App Store submission
\end{enumerate}

\end{document}
